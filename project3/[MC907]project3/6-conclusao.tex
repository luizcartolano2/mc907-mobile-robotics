\section{Conclusions} \label{sec:conclusao}
    In general, the work did not show great results. But, it definitely was great for first experiment with the subject. Also, it was amazing to introduce with the application of machine learning techniques for the visual odometry problem.
    
    Some positive points we can extract from this work is the fact that the using of Deep Learning techniques are plausible to the VO problem, and also, they showed that the simulator images are a good path to make a dataset for the problem.
    
    Weaknesses of the work, which need to be improved in future iterations, are the number of images in the dataset, also some improvements on the neural network, in order to ensure a better fit to the model. Also, we could improve system's effectiveness is to add some kind of SLAM \cite{lim2014real} or a salient model to preprocessing the images.